\documentclass[a4]{article}
\usepackage[utf8]{inputenc}
\usepackage[french]{babel}
\usepackage{listings}
\usepackage{color}
\usepackage{graphicx}
\usepackage[T1]{fontenc}
\usepackage{pdfpages}
\usepackage{geometry}
\geometry{hmargin=2.5cm,vmargin=2.5cm}

\definecolor{mygreen}{rgb}{0,0.6,0}
\definecolor{mygray}{rgb}{0.5,0.5,0.5}
\definecolor{mymauve}{rgb}{0.58,0,0.82}

\lstset{
  backgroundcolor=\color{white},   % choose the background color; you must add \usepackage{color} or \usepackage{xcolor}
  basicstyle=\footnotesize,        % the size of the fonts that are used for the code
  breakatwhitespace=false,         % sets if automatic breaks should only happen at whitespace
  breaklines=true,                 % sets automatic line breaking
  captionpos=b,                    % sets the caption-position to bottom
  commentstyle=\color{mygreen},    % comment style
  deletekeywords={...},            % if you want to delete keywords from the given language
  escapeinside={\%*}{*)},          % if you want to add LaTeX within your code
  extendedchars=true,              % lets you use non-ASCII characters; for 8-bits encodings only, does not work with UTF-8
  frame=L,	                       % adds a frame around the code
  keepspaces=true,                 % keeps spaces in text, useful for keeping indentation of code (possibly needs columns=flexible)
  keywordstyle=\color{blue},       % keyword style
  language=C,                 	   % the language of the code
  otherkeywords={*,...},           % if you want to add more keywords to the set
  numbers=none,                    % where to put the line-numbers; possible values are (none, left, right)
  numbersep=5pt,                   % how far the line-numbers are from the code
  numberstyle=\tiny\color{mygray}, % the style that is used for the line-numbers
  rulecolor=\color{black},         % if not set, the frame-color may be changed on line-breaks within not-black text (e.g. comments (green here))
  showspaces=false,                % show spaces everywhere adding particular underscores; it overrides 'showstringspaces'
  showstringspaces=false,          % underline spaces within strings only
  showtabs=false,                  % show tabs within strings adding particular underscores
  stepnumber=2,                    % the step between two line-numbers. If it's 1, each line will be numbered
  stringstyle=\color{mymauve},     % string literal style
  tabsize=2,	                   % sets default tabsize to 2 spaces
  title=\lstname                   % show the filename of files included with \lstinputlisting; also try caption= instead of title
}
%gestion des caractères latins
\lstset{literate=
  {á}{{\'a}}1 {é}{{\'e}}1 {í}{{\'i}}1 {ó}{{\'o}}1 {ú}{{\'u}}1
  {Á}{{\'A}}1 {É}{{\'E}}1 {Í}{{\'I}}1 {Ó}{{\'O}}1 {Ú}{{\'U}}1
  {à}{{\`a}}1 {è}{{\`e}}1 {ì}{{\`i}}1 {ò}{{\`o}}1 {ù}{{\`u}}1
  {À}{{\`A}}1 {È}{{\'E}}1 {Ì}{{\`I}}1 {Ò}{{\`O}}1 {Ù}{{\`U}}1
  {ä}{{\"a}}1 {ë}{{\"e}}1 {ï}{{\"i}}1 {ö}{{\"o}}1 {ü}{{\"u}}1
  {Ä}{{\"A}}1 {Ë}{{\"E}}1 {Ï}{{\"I}}1 {Ö}{{\"O}}1 {Ü}{{\"U}}1
  {â}{{\^a}}1 {ê}{{\^e}}1 {î}{{\^i}}1 {ô}{{\^o}}1 {û}{{\^u}}1
  {Â}{{\^A}}1 {Ê}{{\^E}}1 {Î}{{\^I}}1 {Ô}{{\^O}}1 {Û}{{\^U}}1
  {œ}{{\oe}}1 {Œ}{{\OE}}1 {æ}{{\ae}}1 {Æ}{{\AE}}1 {ß}{{\ss}}1
  {ű}{{\H{u}}}1 {Ű}{{\H{U}}}1 {ő}{{\H{o}}}1 {Ő}{{\H{O}}}1
  {ç}{{\c c}}1 {Ç}{{\c C}}1 {ø}{{\o}}1 {å}{{\r a}}1 {Å}{{\r A}}1
  {€}{{\EUR}}1 {£}{{\pounds}}1
}
%definition d'un syle pour les documents text
\lstdefinestyle{txt}{
	frame=none,
	numbers=none,
	stringstyle=\color{black},
}

\begin{document}
	\title{\Huge{\textbf{
	\bigbreak
	Manuel d'utilisation}}}
	\author{Dcrypt}
	\date{
	\begin{center}\includegraphics[scale=0.4]{logo.png}\end{center}
	\bigbreak
	\vspace{14cm}
	29 mai 2017
	}
		

	\begin{titlepage}
		\maketitle
		\vspace{20em}
		%\begin{center}\includegraphics{logo_uvsq.jpg}\end{center}
	\end{titlepage}
	\section{Introduction}
L'application Dcrypt est un outil automatique d'aide au decryptage permettant a son utilisateur d'effectuer une cryptanalyse
sur un fichier ou texte à partir d'un ordinateur. Pour cela, le procédé de Vigenère ou de Substitution lui sera proposé. \\
Le client pourra aussi effectuer le cryptage à l'aide de ces 2 methodes ou encore lancer seulement une analyse frequentielle. \\
L’application se veut très simple d’utilisation. Ce guide a été concu afin de repondre à toutes les questions d'utilisation que
pourrait avoir le client.


	\section{Pré-conditions/Materiel necessaire}

		\subsection{Installations requises}
			Certaines installations sont essentielles a l'utilisation de l'application.\\
			Tout d'abord, celle de la bibliotheque GTK: sudo apt-get install libgtk2.0-dev \\
			Enfin, pour Cunit:  sudo apt-get install libcunit1 libcunit1-doc libcunit1-dev \\

		\subsection{Lancement de l'application}
			commande pour lancer appli: make run \\
			commande pour lancer tests: make test\\


	\section{Guide d'utilisation}
	
	
	Ce guide pratique a pour objectif de vous guider dans l’utilisation de l’application
"Dcrypt" et de répondre aux éventuelles questions que vous pourriez vous poser au
cours de son usage. 
	
	

	\section{Menu/Acceuil}
			Ici, vous vous trouvez sur le menu ou page d'acceuil de l'application. \\
			Vous pouvez a l'aide de nos 3 boutons effectuer 3 actions différentes.\\
			Le premier bouton permet le cryptage(on se trouvera alors en (5) ), le second de 
			decrypter un texte (6) et enfin le dernier d'effectuer une analyse frequentielle seule (7).
			\begin{center}\includegraphics[scale=0.4]{1.png}\end{center}
			
			
			
	\section{Menu Cryptage}
		Dans le menu cryptage, il est maintenant possible de choisir le type de cryptage voulu.\\
		Le choix de l'utilisateur peut alors se porter vers un cryptage par Substitution (5.1) ou bien
		vers un cryptage par Vigenère (5.2).\\
		Il lui est egelement possible de revenir en arriere(au menu precedent) à l'aide du bouton "R".
		
		\begin{center}\includegraphics[scale=0.4]{2.png}\end{center}
		\subsection{Cryptage Substitution}
			Vous pouvez maintenant rentrer/taper votre texte à l'aide de la zone d'entrée (8)
			ou bien le charger à l'aide de la fenetre de chargement (9).
			\begin{center}\includegraphics[scale=0.4]{3.png}\end{center}
			Le resultat sera maintenant affiché. \\
			Vous pouvez maintenant enregistrer la clé et le texte chiffré à l'aide 
			de la fenetre d'enregistrement (9)
			\begin{center}\includegraphics[scale=0.4]{5.png}\end{center}
		\subsection{Cryptage Vigenere}
		Dans ce menu on vous demande tout dabord d'entrer la clé de cryptage.\\
		Cette derniere ne doit contenir que des lettres et sa taille doit etre inferieure ou egale a 12.
			\begin{center}\includegraphics[scale=0.4]{22.png}\end{center}
			A present, vous devez indiquer le texte que vous voulez crypter. Vous pouvez soit
			le rentrer directement (8) ou bien le charger (9).
			\begin{center}\includegraphics[scale=0.4]{6.png}\end{center}
			Le resultat sera alors affiché et il est possible pour l'utilisateur 
			d'enregistrer la clé utilisée et le texte maintenant crypté (9).
			\begin{center}\includegraphics[scale=0.4]{7.png}\end{center}
		
		
	\section{Menu Decryptage}
		Avant tout, vous devez indiquer la langue du texte que vous voulez decrypter. \\
		Les ressources utilisées pour le decryptage ne sont pas les memes en francais et en anglais.
		\begin{center}\includegraphics[scale=0.4]{8.png}\end{center}
		Dans le menu decryptage, vous pouvez soit choisir le type de decryptage (methode de Vigenere ou Substitution) ou bien revenir au menu principal à l'aide du bouton "R".
		\begin{center}\includegraphics[scale=0.4]{9.png}\end{center}
		\subsection{Decryptage Substitution}
			Vous devez desormais rentrer votre texte (8) ou bien le charger (9).
			\begin{center}\includegraphics[scale=0.4]{10.png}\end{center}
			Maintenant vous avez une partie de la clé qui est decryptée et vous allez 
			essayer de trouver quelques autres caracteres. \\
			\textit{Exemple :}\\
			Si le resultat est "kONvOUR...", vous pouvez etre sur que k correspond a B et v a J. 
			\begin{center}\includegraphics[scale=0.4]{13.png}\end{center}
			Vouq pouvez faire les modifications avec le menu "changer clé". 
			il manque un screen ICI
			au bout de 4 ou 5 caractéres retrouvés le texte sera decrypté (car plus de la moitié 
			des caracteres est decryptée par l'application) et vous aurez:
			\begin{center}\includegraphics[scale=0.4]{11.png}\end{center}
			Vous pouvez cliquer sur "terminer" qui vous emmenera sur ce menu qui vous permet comme pour 
			les cryptages d'enregistrer le resultat et la clée.
			\begin{center}\includegraphics[scale=0.4]{12.png}\end{center}
		\subsection{Decryptage Vigenere}
			La vous pouvez soit rentrer votre texte a l'aide de la zone d'entrée(8)
			soit le charger a l'aide de la fenetre de chargement (9).
			\begin{center}\includegraphics[scale=0.4]{14.png}\end{center}
			La vous avez une cle et un texte decrypté; normalement la clé obtenu est la bonne.
			Si le texte est assez long si le texte est illisible alors la clé est incorrecte et
			 dans ce cas il suffit de choisir le bouton "redechiffrer" qui va ressayer avec une 
			 autre taille de clé. Une fois le resultat trouvé vous pouvez a l'aide du bouton 
			 terminer aller au menu de resultat final.  
			\begin{center}\includegraphics[scale=0.4]{15.png}\end{center}
			Ce menu vous permet comme pour 
			les cryptages d'enregistrer le resultat et la clé.
			\begin{center}\includegraphics[scale=0.4]{16.png}\end{center}
		
		
	\section{Menu Analyse Frequentielle}
			Afin d'effectuer l'analyse frequentielle sur un texte il faut evidemment le renseigner.
			Pour se faire, l'utilisateur peut le rentrer directement (8) ou bien charger le texte (9).
		\begin{center}\includegraphics[scale=0.4]{17.png}\end{center}
			Présentement, vous aurez les frequences dans le texte de chaque lettre de l'alphabet ainsi que
			celle de chaque digramme et trigramme de lettres.\\
			Remarque: Si le texte est trop long, elle n'affichera que ceux qui se repetent le plus.
		\begin{center}\includegraphics[scale=0.4]{18.png}\end{center}
	\section{Entrer un texte}
		A chaque fois que vous allez essayer de rentrer un texte cette fenetre s'affichera devant vous.
		\begin{center}\includegraphics[scale=0.4]{21.png}\end{center}
	\section{les Chargements/Enregistrement de fichiers}
		A chaque fois que vous allez essayer de charger un texte cette fenetre aparaitra.
		\begin{center}\includegraphics[scale=0.4]{19.png}\end{center}
		A chaque fois que vous allez essayer d'enregistrer un texte cette fenetre s'affichera devant vous.
		\begin{center}\includegraphics[scale=0.4]{20.png}\end{center}
\end{document}
